\chapter*{Conclusion}
 \addcontentsline{toc}{chapter}{Conclusion}
 \adjustmtc
\renewcommand\partie{\Partie\ CONCLUSION}
\label{ch-35}

\bigskip
\minitoc  

Cette Partie \ref{part_3} contient l'état actuel de notre étude numérique de l'effet de l'anisotropie de pression sur la cascade turbulente. 

Dans le Chapitre \ref{ch-31}, nous présentons le code qui nous a permis d'obtenir les données dans lesquelles l'étude des lois exactes est effectuée ainsi que notre méthode de post-traitement. Cette méthode reposant sur l'usage de la transformée de Fourier, n'est à notre connaissance pas utilisée par la communauté. 

Dans le Chapitre \ref{ch-32}, sont exposées les étapes ayant permis la validation de notre code ainsi qu'une étude de l'apport de notre méthode sur une méthode utilisée couramment consistant à décrire l'espace des échelles par un ensemble réduit de vecteurs. En se basant sur nos connaissances du code et sur le travail analytique de la Partie \ref{part_2}, nous y développons une méthode d'obtention de l'erreur numérique s'appliquant sur nos résultats ainsi qu'une analyse approfondie de nos sources d'erreur. 

Ainsi armé de ces outils, nous avons attaqué l'analyse complète des simulations utilisées par \cite{ferrand_fluid_2021} afin de valider l'apport de notre extension gyrotrope de la théorie des lois exactes et d'affiner notre compréhension de l'effet de l'anisotropie de pression sur la turbulence. Le Chapitre \ref{ch-33} contient une analyse préliminaire des simulations du modèle \acs{CGLHPe}. Cette étude valide l'apport de notre extension en particulier le poids, dans des simulations incompressibles, du terme survivant dans la limite incompressible et qui a fait l'objet du Chapitre \ref{ch-22}. Cependant, l'interprétation de ses résultats est encore sujette à discussion et nécessitera quelques analyses complémentaires. Le Chapitre \ref{ch-34} contient une ouverture vers l'application de nos résultats analytiques et de nos méthodes dans des simulations plus complexes prenant en compte l'effet Landau-fluide. De telles simulations rapprochent le comportement du système de celui décrit par un modèle cinétique en captant partiellement l'effet Landau linaire. Les tout premiers résultats questionnent notre interprétation de l'impact du flux de chaleur sur la cascade turbulente, et nécessiteront une analyse plus fine avant de mener à un début d'interprétation. 




