\chapter*{Introduction}
 \addcontentsline{toc}{chapter}{Introduction}
 \adjustmtc
\renewcommand\partie{\Partie\ INTRO}
\label{ch-10}

Le modèle incompressible est encore très utilisé [\cite{marino_scaling_2023}] pourtant le caractère compressible des fluctuations et des structures présentes dans le vent solaire est observé et identifié depuis les premières missions spatiales [\cite{tu_mhd_1995}]. Les travaux présentés dans cette partie se placent dans la continuité d'un effort d'extension de la théorie de Kolmogorov aux plasmas compressibles entrepris depuis \cite{banerjee_exact_2013}. 

Dans le Chapitre \ref{ch-12}, sera présenté le modèle compressible sur lequel seules deux contraintes seront imposées dans cette Partie \ref{part_1} : une pression de type isotrope et une équation d'induction \ac{MHD}. Diverses relations thermodynamiques y seront analysées pour fermer ce modèle fluide. 

Dans le Chapitre \ref{ch-13}, sera résumé l'extension analytique compressible avec pression isotrope de la théorie de Kolmogorov à travers les premiers résultats que j'ai obtenus. 

Dans le Chapitre \ref{ch-14}, nous parlerons observation et données in-situ à travers une application de nos premiers résultats analytiques dans les premières données obtenues près du Soleil par \ac{PSP}. Cette étude de cas effectuée, j'ai ensuite amorcé une étude statistique dans des données relevées dans la magnétogaine terrestre par \ac{MMS}. 

         
