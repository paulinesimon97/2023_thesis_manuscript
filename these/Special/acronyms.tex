% More infor at http\string://ctan.mines-albi.fr/macros/latex/contrib/acronym/acronym.pdf


\begin{acronym}
  
   \acro{1D}{1 dimension\acroextra{.}}
   \acro{2D}{2 dimensions\acroextra{.}  }
   \acro{3D}{3 dimensions\acroextra{.}  }
   \acro{FFT}{algorithme de transformée de Fourier rapide\acroextra{.}}
   

    \acro{KHM}[KHM]{[\cite{von_karman_statistical_1938,monin_statistical_1971}]\acroextra{. Type de loi exacte obtenue avant applications des hypothèses de stationnarité statistique et séparation d'échelle.}}
    
    \acro{K41}[K41]{[\cite{kolmogorov_local_1991,kolmogorov_dissipation_1991}]\acroextra{.Type de loi exacte obtenue après applications des hypothèses de stationnarité statistique et séparation d'échelle.}}
    \acro{PP98}{[\cite{politano_dynamical_1998,politano_von_1998}]\acroextra{. Loi exacte MHD incompressible.}}
    \acro{F21}{[\cite{ferrand_fluid_2021}]}
    \acro{A18}{[\cite{andres_energy_2018}]}
    \acro{BG17}{[\cite{banerjee_exact_2017}]}
    \acro{FEL}{<<Fourier for Exact Law>>}

    \acro{MHDHPe}[MHD-Hall-\ensuremath{\nabla P_e}]{Modèle prenant en compte des corrections Hall et \ensuremath{\nabla P_e} de la loi d'Ohm.}
    \acro{MHDH}[MHD-Hall]{Modèle prenant en compte la correction Hall de la loi d'Ohm.}
    \acro{IMHD}[IMHD]{Modèle MHD incompressible\acroextra{.}}
    \acro{IMHDH}[IMHD-Hall]{Modèle MHD incompressible prenant en compte la correction Hall de la loi d'Ohm}
    \acro{HD}{hydrodynamique\acroextra{.}}
    \acro{MHD}{magnétohydrodynamique\acroextra{. Modèle et zone d'échelle.}}
    \acro{Hall}[Hall]{\acroextra{. Gamme d'échelle subionique et terme de la loi d'Ohm.}}
    \acro{Pe}[\ensuremath{\nabla P_e}]{terme dépendant de la pression électronique et présent dans la loi d'Ohm généralisée}
    \acro{Pi}[\ensuremath{\nabla P_i}]{Terme dépendant de la pression ionique présent dans la loi d'Ohm généralisée valable en régime \acs{EMHD}.}
    \acro{CGL}{[\cite{chew_boltzmann_1956}]\acroextra{. Modèle gyrotrope et fermeture.}}
    \acro{CGLH}[CGL-MHD-Hall]{Modèle prenant en compte l'approximation Hall et la fermeture CGL.}
    \acro{LFCGLHPe}[LF/CGL-MHD-Hall-\ensuremath{\nabla P_e}]{Modèle versatile simulé.}
    \acro{CGLHPe}[CGL-MHD-Hall-\ensuremath{\nabla P_e}]{Modèle CGL simulé.}
    \acro{LFHPe}[LF-Hall-\ensuremath{\nabla P_e}]{Modèle Landau-fluide simulé.}
    \acro{LF}{Landau-fluide\acroextra{. Modèle gyrotrope et fermeture.}}
    \acro{EMHD}{magnétohydrodynamique électronique\acroextra{. Modèle et zone d'échelle couvrant les échelles sub-ioniques et électroniques.}}
    
    \acro{NASA}{National Aeronautics and Space Administration\acroextra{. Agence spatiale américaine.}}
    \acro{ESA}{European Space Agency \acroextra{. Agence spatiale européenne.}}
    \acro{PSP}{Parker Solar Probe\acroextra{. Mission spatiale de la \ac{NASA} constituée d'une sonde.}}
    \acro{MMS}{Magnetospheric Multiscale\acroextra{. Mission spatiale de la \ac{NASA} constituée de quatre sondes.}}
    \acro{SWEAP}{Solar Wind Electrons Alpha and Protons Investigation}
    \acro{FIELDS}{Fields Experiments}
    \acro{MAGs}{magnétomètres à saturation présent sur PSP}
    \acro{SCM}{fluxmètre présent sur PSP}
    \acro{SPC}{coupe de Faraday présent sur PSP}
    \acro{SPAN}{analyseur électrostatiques présent sur PSP}
    \acro{UTC}{Temps Universel Coordonné}
    \acro{RTN}{système de coordonnées local de la position du satellite tel que \ensuremath{e_z} est tourné vers le centre de l'objet autour duquel le satellite orbite. }
    \acro{FPI}{Fast Plasma Investigation sur MMS}
    \acro{FGM}{Fluxgate Magnetometer sur MMS}
    \acro{WIND}{satellite présent dans le vent solaire\acroextra{. Mission spatiale de la \ac{NASA} constituée d'une sonde.}}
    \acro{CLUSTER}{quatre satellites pour l'étude de la magnétosphère\acroextra{. Mission spatiale de l'\ac{ESA} constituée de quatre sondes.}}
    \acro{THEMIS}{Time History of Events and Macroscale Interactions during Substorms\acroextra{. Mission spatiale de la \ac{NASA} constituée de cinq satellites.}}
    \acro{JUICE}{Jupiter Icy Moons Explorer}
    \acro{CME}{éjections de masse coronale}
    \acro{ACE}{Advanced Composition Explorer}
    
  \acro{LPP}{Laboratoire de Physique des Plasmas\acroextra{. \acf{UMR} 7648. Situé à Paris et Palaiseau.} }
  \acro{SU}{Sorbonne Université\acroextra{.} }
  \acro{IAP}{Institut d'Astrophysique de Paris\acroextra{. \acf{UMR} 7095. Situé à Paris.} }
  \acro{OBSPM}{Observatoire de Paris-Meudon\acroextra{. Situé à Paris et Meudon.} }
  \acro{LAGRANGE}{Laboratoire J.-L. LAGRANGE\acroextra{. \acf{UMR} 7293. Situé à Nice.}}
  \acro{OCA}{Observatoire de la Côte d'Azur\acroextra{. Situé à Nice.}}
  \acro{CNRS}{Centre national de la recherche scientifique\acroextra{. Situé sur l'ensemble du territoire français.}}
  \acro{UMR}{Unité Mixte de Recherche\acroextra{. Système français de classification des laboratoires.}}
  \acro{CNR}{Consiglio Nazionale delle Ricerche\acroextra{. Centre national italien pour la recherche.}}
  \acro{ISTP}{Istituto per la Scienza e la Tecnologia dei Plasmi\acroextra{. Situé à Milan, Bari et Padova.}}
  \acro{IRAP}{Institut de recherche en astrophysique et planétologie\acroextra{. Situé à Toulouse.}}
  \acro{Imperial}{Imperial College London\acroextra{. Situé à Londres.}}
  \acro{SPAT}{Space and Atmospheric Physics group\acroextra{. Sous composante de \ac{Imperial}}}
  \acro{Univ.}{université}
  \acro{AAIF}{Ecole doctorale n$^{\circ}$127 : Astronomie et Astrophysique d'Ile-de-France\acroextra{.} }
  \acro{UPSy}{Université Paris-Saclay\acroextra{.}}
  \acro{X}{École Polytechnique\acroextra{. Situé à Palaiseau.}}
  % \acro{BC}{Boundary Conditions }
  
  % \acro{CNES}{Centre National d'Etude Spatial}
  
  % \acro{DFT}{Transformation de Fourier Discrète}
  % \acro{FT}{Fourier Transform}
  % \acro{FFT}{Fast Fourier Transform}
  
  % \acro{DR}{Dispersion Relation \acroextra{. Relation between the wave number and complex frequency for waves in plasmas.}}
  
  % \acro{IEDF}{Ionic Energy Distribution Function }
  % \acro{IEPF}{Ionic Energy Probability Function }
  % \acro{IVDF}{Ionic Velocity Distribution Function }
  
  % \acro{IAW}{Ion Acoustic Wave}
 
  % \acro{LHS}{Left Hand Side }
  % 

  % \acro{PIC}{Particle In Cell}
  
  % \acro{RMS}{Root Mean Square}

\end{acronym}
