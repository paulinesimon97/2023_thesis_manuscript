% résumé en anglais (4 000 caractères maximum) 
	The solar wind is a highly turbulent plasma in which the fluid quantities (fluid velocity, density and pressure) and the electric and magnetic field vary greatly. These fluctuations are reflected in turbulent spectra covering several frequency decades. The spectra of velocity and magnetic field fluctuations follow power-laws whose exponents depend on the scale considered. At low frequencies, exponents close to -5/3 are observed. They are interpreted as a signature of magnetohydrodynamic (MHD) turbulence that transfers energy non-linearly from large scales to small scales, where dissipation is possible.
	
	This turbulent cascade can be studied using exact laws derived from the equations of the fluid quantities and the magnetic field. These laws link the cascade rate to the turbulent fluctuations. The cascade rate is associated with the dissipation rate according to Kolmogorov's theory, and would therefore provide an estimation of the plasma heating. This dissipation rate is a key to understanding the problem of solar wind heating. Indeed, the temperature of the solar wind decreases slower (with heliocentric distance) than predicted by the theory of adiabatic radial expansion.  
	
   	 In recent years, the theory of exact laws has been successfully extended to MHD and Hall-MHD models, with the Hall effect extending the range of validity of MHD to scales comparable to or smaller than the characteristic ion scale. These laws have been derived within the framework of the incompressible (constant density) and isothermal (pressure proportional to density) approximations. These closures simplify the equations describing the plasma. However, their validity as a description of space plasmas such as the solar wind is open to question. A bit more realistic hypothesis would be to take into account a polytropic closure (pressure proportional to a power of the density). By extending the theory of exact laws in this direction, a more versatile law was obtained: it depends on any isotropic (scalar) pressure. 
   	 
The contribution of this law is then analysed through an application to data collected in the solar wind by the Parker Solar Probe launched by NASA in 2018 to explore the Sun.
	In space plasmas, the magnetic field and the lack of collisions induce pressure anisotropy.  Such pressure takes the form of a full tensor and can be reduced to the gyrotropic one that considers at least the difference between the components parallel and perpendicular to the ambient magnetic field.  A new extension of the theory of exact laws is then derived, relaxing the pressure isotropy assumption. The law obtained is applicable to flows governed by a tensorial pressure and described, at least, by a CGL (Chew, Goldberger, Low, 1956) closure, also known as bi-adiabatic because it depends on pressure gyrotropy. This law provides a rigorous framework for studying the impact of pressure anisotropy on the turbulent cascade and the heating rate. In order to validate its contribution and refine its interpretation, the CGL version of the law was finally applied to three-dimensional turbulent simulations of the Hall-MHD-CGL model.
