% résumé en français (4 000 caractères maximum)
	Le vent solaire est un plasma hautement turbulent où les quantités fluides (vitesse, densité et pression) ainsi que les champs électrique et magnétique varient beaucoup. Ces fluctuations se traduisent entre autres par des spectres turbulents couvrant plusieurs décades en fréquences. Les spectres des fluctuations de vitesse et de champ magnétique suivent des lois de puissances dont les exposants dépendent de l’échelle considérée. Aux basses fréquences, des exposants proches de -5/3 sont observés. Ils sont interprétés comme une signature de la dynamique turbulente magnétohydrodynamique (MHD) du plasma. Cette dynamique prendrait la forme d’une cascade transférant non-linéairement l’énergie présente à grande échelle, vers les petites échelles où sa dissipation est possible.
	
	Cette cascade turbulente peut être étudiée au moyen de lois exactes dérivées à partir des équations des différentes quantités fluides. Ces lois lient le taux de cascade aux fluctuations turbulentes. Le taux de cascade est associé au taux de dissipation d’après la théorie de Kolmogorov et correspondrait donc à l’estimation d’un taux de chauffage du plasma. Ce taux est une clef de compréhension du problème du chauffage du vent solaire, la température de ce dernier décroissant plus lentement avec la distance héliocentrique que ne le prédit la théorie de l’expansion radiale adiabatique.
	
	Ces dernières années, la théorie des lois exactes a été étendue avec succès aux modèles MHD et MHD-Hall, l’effet Hall étendant le domaine de validité de la MHD à des échelles comparables ou plus petites que l’échelle caractéristique des ions. Ces lois ont été dérivées dans le cadre d’approximations (fermetures) de type incompressible (densité constante) ou isotherme (pression proportionnelle à la densité). Ces fermetures permettent de simplifier les équations décrivant le plasma. Cependant, leur validité est sujette à caution dans le cadre des plasmas spatiaux tels que le vent solaire. Une hypothèse un peu plus réaliste consisterait à prendre en compte une fermeture du type polytrope (pression proportionnelle à une puissance de la densité).  En étendant la théorie des lois exactes dans cette direction, une loi plus versatile a été obtenue : elle dépend d’une pression isotrope (scalaire) quelconque.  L’apport de cette loi a ensuite été analysé à travers une application à des données relevées dans le vent solaire par la sonde Parker Solar Probe lancée par la NASA en 2018 en direction du Soleil.
	
	Dans les plasmas spatiaux, il s’avère que le champ magnétique et le manque de collisions induisent une anisotropie de pression. La pression est alors tensorielle et prend en compte, a minima, une différence de pression parallèlement et perpendiculairement au champ magnétique ambiant (hypothèse gyrotrope).  Une nouvelle étape d’extension de la théorie des lois exactes a donc été entreprise en relaxant l’isotropie de la pression. La loi obtenue est applicable à des écoulements régis par une pression tensorielle décrits par exemple par une fermeture CGL (Chew, Goldberger, Low, 1956), dite aussi bi-adiabatique, car dépendant de la gyrotropie de pression. Cette loi apporte un cadre d’étude rigoureux de l’impact de l’anisotropie sur la cascade turbulente et le taux de chauffage. Afin de valider son apport et d’affiner son interprétation, la loi CGL est enfin appliquée dans des simulations tridimensionnelles turbulentes du modèle CGL-MHD-Hall.
