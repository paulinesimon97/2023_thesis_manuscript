% More infor at http\string://ctan.mines-albi.fr/macros/latex/contrib/acronym/acronym.pdf


\begin{acronym}
  
  \defacro{HD}{HD}{hydrodynamique}{} 
  \defacro{MHD}{MHD}{magnétohydrodynamique}{Modèle et gamme d'échelle.}
  \defacro{IMHD}{Inc-MHD}{MHD incompressible}{}
  \defacro{EMHD}{EMHD}{MHD électronique}{Modèle et zone d'échelle couvrant les échelles sub-ioniques et électroniques.}
  \defacro{Hall}{Hall}{Hall}{Gamme d'échelle subionique et terme de la loi d'Ohm.}
  \defacro{MHDH}{MHD-Hall}{MHD-Hall}{Modèle prenant en compte la correction Hall de la loi d'Ohm.}
  \defacro{IMHDH}{Inc-MHD-Hall}{MHD-Hall incompressible}{Modèle MHD incompressible prenant en compte la correction Hall de la loi d'Ohm.}  
  \defacro{MHDHPe}{MHD-Hall-\ensuremath{\nabla P_e}}{MHD-Hall avec correction \ensuremath{\nabla P_e}}{Modèle prenant en compte des corrections Hall et \ensuremath{\nabla P_e} de la loi d'Ohm.}
  \defacro{Pe}{\ensuremath{\nabla P_e}}{\og grad Pe \fg{}}{Terme dépendant de la pression électronique présent dans la loi d'Ohm généralisée.}
  \defacro{Pi}{\ensuremath{\nabla P_i}}{\og grad Pi \fg{}}{Terme dépendant de la pression ionique présent dans la loi d'Ohm généralisée valable en régime \sacro{EMHD}.}
  \defacro{CGL}{CGL}{Chew-Goldberger-Low}{[\cite{chew_boltzmann_1956}]. Modèle et fermeture dépendant d'une pression gyrotrope (bi-adiabatique).}
  \defacro{CGLH}{CGL-MHD-Hall}{MHD-Hall bi-adiabatique}{Modèle prenant en compte l'approximation Hall et la fermeture CGL.}
  \defacro{LF}{LF}{Landau-fluide}{Modèle et fermeture gyrotrope captant l'effet Landau cinétique linéaire.}
    
  
\end{acronym}
