% More infor at http\string://ctan.mines-albi.fr/macros/latex/contrib/acronym/acronym.pdf


\begin{acronym}
  \defacro{NASA}{NASA}{National Aeronautics and Space Administration}{Agence spatiale américaine.}
  \defacro{ESA}{ESA}{European Space Agency}{Agence spatiale européenne.}
  \defacro{PSP}{PSP}{Parker Solar Probe}{Mission spatiale de la \sacro{NASA} constituée d'une sonde.}
  \defacro{MMS}{MMS}{Magnetospheric Multiscale}{Mission spatiale de la \sacro{NASA} constituée de quatre sondes.}
  \defacro{SWEAP}{SWEAP}{Solar Wind Electrons Alpha and Protons Investigation}{}
  \defacro{FIELDS}{FIELDS}{Fields Experiments}{}
  \defacro{MAGs}{MAGs}{MAGs}{magnétomètres à saturation présent sur PSP.}
  \defacro{SCM}{SCM}{Search-Coil Magnetometer}{Fluxmètre présent sur PSP.}
  \defacro{SPC}{SPC}{Solar Probe Cup}{Coupe de Faraday présent sur PSP.}
  \defacro{SPAN}{SPAN}{Solar Probe Analyzer}{Analyseur électrostatiques présent sur PSP.}
  \defacro{UTC}{UTC}{Temps Universel Coordonné}{}
  \defacro{RTN}{RTN}{système de coordonnées Radial-Tangential-Normal}{Système de coordonnées local de la position du satellite tel que \ensuremath{e_z} est tourné vers le centre de l'objet autour duquel le satellite orbite.}
  \defacro{FPI}{FPI}{Fast Plasma Investigation}{Sur MMS.}
  \defacro{FGM}{FGM}{Fluxgate Magnetometer}{Sur MMS.}
  \defacro{WIND}{WIND}{Wind}{Satellite présent dans le vent solaire. Mission spatiale de la \sacro{NASA} constituée d'une sonde.}
  \defacro{CLUSTER}{CLUSTER}{Cluster}{Quatre satellites pour l'étude de la magnétosphère. Mission spatiale de l'\sacro{ESA} constituée de quatre sondes.}
  \defacro{THEMIS}{THEMIS}{Time History of Events and Macroscale Interactions during Substorms}{Mission spatiale de la \sacro{NASA} constituée de cinq satellites.}
  \defacro{JUICE}{JUICE}{Jupiter Icy Moons Explorer}{}
  \defacro{CME}{CME}{éjections de masse coronale}{}
  \defacro{ACE}{ACE}{Advanced Composition Explorer}{}
  \defacro{HELIOSWARM}{HELIOSWARM}{HelioSwarm}{Future mission spatiale de la \sacro{NASA} constituée de neuf sondes.}
  
  
\end{acronym}
