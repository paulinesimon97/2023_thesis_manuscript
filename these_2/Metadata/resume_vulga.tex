	Le Vent Solaire (VS) est un gaz ionisé et magnétisé, éjecté par le Soleil dans l’espace interplanétaire. La pression n’y est pas définie par une seule valeur, mais par au moins deux valeurs, suivant si l’on regarde le plasma dans la direction du champ magnétique moyen ou perpendiculairement à cette direction. Cette anisotropie de pression impacte le plasma et peut même le rendre instable. Le VS est aussi un milieu turbulent. La turbulence engendre une cascade d’énergie des grandes vers les petites échelles, permettant ainsi un chauffage du plasma. Dans ce travail, nous posons la question de l’impact de l’anisotropie de pression sur la turbulence d’un plasma tel que le VS. La première étape est analytique et étend, à tout type de pression, la théorie des lois exactes de Kolmogorov, qui permet une estimation du taux de cascade. Puis une étude numérique est menée afin de comprendre l’apport de cette extension à notre interprétation de la dynamique turbulente.