	The Solar Wind (SW) is a plasma, an ionized gas that interacts with the magnetic and electric fields and is ejected by the Sun, in the interplanetary space. In this medium, the pressure is not defined as a single quantity, but rather by, at least, two quantities, parallel and perpendicular to the magnetic field. Such anisotropy impacts the plasma that can become unstable. The SW is also a turbulent fluid. The turbulence results in a cascade of energy from large scales to small ones, where the energy can heat the plasma. In this work, we investigate the impact of the pressure anisotropy on the turbulence of the SW. The first step is analytical and extends the theory of Kolmogorov that gives an estimation of the turbulent cascade rate, to all types of pressure. Then a numerical study is led to understand the impact of this extension in our interpretation of a turbulent behaviour.