Depuis 1998 et la loi exacte \cacro{PP98} étendant aux plasmas \cacro{IMHD} la théorie de Kolmogorov décrivant la cascade turbulente à travers des lois exactes, de multiples extensions ont été proposées prenant en compte la compressibilité. 

Dans cette partie \ref{part_1}, nous nous sommes concentrés sur l'effet de fermetures thermodynamiques dépendant d'une pression isotrope. Un premier chapitre (synthèse section \ref{synt-12}) pose le problème de la compressibilité dans les modèles fluides et analyse différentes possibilités de fermeture basées sur la théorie thermodynamique. La question qui se pose alors est celle de l'impact de la compressibilité sur la turbulence. Ma contribution pour y répondre est développée à travers les chapitres \ref{ch-13} et \ref{ch-14}. 

Dans le Chapitre \ref{ch-13} (synthèse section \ref{synt-13}), un cadre analytique est démontré à travers l'extension de la théorie des lois exactes. La stratégie mise en œuvre ne repose pas sur une fermeture thermodynamique, contrairement à celles entreprises dans la littérature [\cite{galtier_exact_2011}, \cite{banerjee_exact_2013}, \cite{banerjee_kolmogorov-like_2014}, \cite{andres_alternative_2017}], mais plutôt sur l'équation de densité d'énergie interne. La loi exacte résultante obtient ainsi un caractère général et la fermeture ne devient qu'un \og détail \fg{}, une hypothèse à ne considérer qu'à la fin du calcul en fonction du besoin. Par ce biais, est abordé l'objectif initial de cette partie du travail : obtenir une loi valable dans la zone inertielle isentrope pour une fermeture polytrope décrivant ainsi la cascade turbulente dans les plasmas de manière plus réaliste et versatile que la fermeture isotherme utilisée jusqu'à présent. La première formulation (f1) proposée pour répondre à cet objectif est inspirée du travail dans le cadre isentrope-isotherme de \cite{andres_energy_2018}. Elle a permis l'étude comparative, dans deux jeux de données issus de la mission \cacro{PSP}, de l'impact de la compression et des fermetures isentropes-isotherme ($\gamma = 1$) et isentrope-adiabatique ($\gamma = 5/3$) sur la cascade turbulente. Cette étude fait l'objet du Chapitre \ref{ch-14}  (synthèse section \ref{synt-14}) où elle est étendue dans la magnétogaine à travers l'amorçage d'une étude statistique utilisant les données de \cacro{MMS}. L'intérêt de la formulation f1 est que les termes sources impossibles à calculer à cause des caractéristiques de la mission \cacro{PSP} (une seule sonde) ont préalablement été numériquement démontrés comme négligeables dans le taux de cascade total par \cite{andres_energy_2018} dans les cadre isotherme. La deuxième formulation (f2) de la loi exacte a initialement vu le jour comme une conséquence du travail analytique qui sera présenté dans la partie \ref{part_2}, relaxant l'isotropie de pression. Ce résultat, dépendant de $p/\rho$, peut s'avérer plus adapté à l'application d'une fermeture thermodynamique. La troisième et dernière formulation (f3) est la plus récente et s'inspire du travail sur le flux de chaleur dans l'équation d'énergie interne qui s'est révélé nécessaire lors de l'étude numérique qui sera présentée dans la partie \ref{part_3}. Ce résumé des résultats obtenus avec pression isotrope reflète la structure chronologique de l'ensemble du travail effectué et présenté dans ces trois parties, la méthode scientifique mise en œuvre et les points méthodologiques utilisés. 

En termes de physique, cette partie propose un cadre d'étude de l'impact de la compression dans sa forme la plus \og simple \fg{} : une densité variable, une pression isotrope, une énergie interne et un flux de chaleur souvent négligé. Ces grandeurs nous permettent de fermer le modèle fluide par des relations basées sur des hypothèses thermodynamiques telles que l'isentropie, l'isothermie ou la polytropie. À travers l'analyse de ces hypothèses et leur application dans les anciennes descriptions de la cascade turbulente, quatre possibilités majeures de fermeture ont émergé. La première, isentrope-isotherme, est la première à avoir été utilisée dans l'extension des lois exactes [\cite{galtier_exact_2011,banerjee_exact_2013,andres_alternative_2017}]. La deuxième, isentrope-polytrope, introduite en \cacro{HD} [\cite{banerjee_kolmogorov-like_2014}], est celle qui nous a permis de généraliser la méthode d'obtention des lois exactes à toutes fermetures en utilisant l'équation d'énergie interne, elle prend en compte l'existence d'un $\gamma$ et reflète un peu mieux la pluralité de transformations thermodynamiques observée dans les plasmas spatiaux et astrophysiques. La troisième, polytropique, basée sur un $\gamma$ et un $\sigma$, lie le flux de chaleur au travail de pression et étend un peu plus loin les possibilités d'application des lois exactes. De la dernière, isotherme, émerge la loi exacte compressible qui semble la plus simple malgré la prise en compte des flux de chaleur. 

Concernant l'impact de la compression et des fermetures observé dans l'étude du taux de cascade dans le vent solaire, l'étude de cas comparative montre que la compression peut jouer un rôle important dans la cascade, mais, dans les cas étudiés, la fermeture isentrope-adiabatique ou isentrope-isotherme a peu d'impact malgré le rôle qu'elle joue à travers l'énergie interne. Les termes dominants s'avèrent en effet être ceux n'en dépendant pas. On peut aussi les interpréter comme ceux subsistant dans le cas d'une fermeture isobare. Ce travail pose ainsi les bases d'une étude observationnelle, plus générale, complète et systématique, de l'impact de la compression et des fermetures sur la turbulence dans les plasmas spatiaux. Cette étude est laissée au futur car le vent solaire ayant la particularité d'être peu collisionnel et magnétisé, nous nous sommes intéressés à un autre type de fermeture qui a orienté le travail dans une autre direction : celle de l'effet de l'anisotropie de pression.  










