Dans cette partie, nous avons dérivé un cadre d'étude complet et rigoureux des écoulements turbulents (zone inertielle supposée isentrope) allant du régime mono-fluide au régime bi-fluide, et dépendant de pressions tensorielles. 

Dans le Chapitre \ref{ch-21} (synthèse \ref{synt-21}), a été présentée une extension de la théorie de Kolmogorov à un écoulement magnétisé idéal dépendant d'une pression tensorielle. Un tenseur de pression gyrotrope a ensuite été appliqué dans cette extension afin de répondre analytiquement à la question de l'impact des anisotropies de pression décrites par le modèle \ac{CGL} sur la cascade turbulente. De nouveaux termes pouvant nourrir ou réduire la cascade et dépendant de $1-a_p$ avec $a_p$ le taux d'anisotropie $p_{\perp}/p_{\parallel}$ ont été découverts. Linéairement, le signe de $1-a_p$ impacte l'apparition d'instabilité firehose ou miroir dans l'écoulement, comme cela a été rappelé dans la section \ref{sec-212}. On a donc proposé un modèle théorique non linéaire permettant d'étudier le lien entre anisotropies (et potentiellement instabilités) et régimes turbulents, qui pourrait venir expliquer les observations de \cite{osman_proton_2013} et \cite{hadid_compressible_2018} dans le vent solaire. Cependant, une étude numérique est nécessaire afin d'affiner l'interprétation de la loi exacte. 

Parmi les nouveaux termes dépendant de l'anisotropie de pression émergeant dans la loi exacte, un terme source survit dans la limite incompressible. La question de sa signification s'est donc posée. Nous nous sommes alors demandés à quoi pourrait ressembler un modèle incompressible gyrotrope dans le Chapitre \ref{ch-22} (synthèse \ref{synt-22}). Un tel modèle incompressible, fermé par l'équation sur la trace du tenseur de pression, a alors été proposé et linéarisé. En plus de l'onde d'Alfvén-firehose, un nouveau mode y apparaît. On y retrouve le critère d'instabilité firehose parallèle mais aussi un critère d'instabilité que l'on a nommé pseudo-firehose apparaissant dans le cas quasi-perpendiculaire et venant réduire la zone de stabilité du modèle en fonction du taux d'anisotropie moyen $a_{p0}$ et du paramètre $\beta_{\parallel 0}$. Après une étude plus fine en fonction de l'angle de propagation, ce mode s'est révélé instable pour certains angles de propagation et tout couple de paramètres $\beta_{\parallel 0}$ et $a_{p0}$ tels que $\frac{\beta_{\parallel 0}}{2}\left(1-a_{p0}\right) \neq \frac{3}{4}$. 
%Dans la limite incompressible du modèle \ac{CGL}, ce mode apparaît sous un format surcontraint. La correction anisotrope de la loi \acs{PP98} refléterait donc l'impact non linéaire de la correction firehose au mode d'Alfvén non linéaire. 
Cette étude fera l'objet d'un futur article. 

Enfin, dans le Chapitre \ref{ch-23} (synthèse \ref{synt-23}), les corrections provenant de la relaxation des approximations appliquées sur l'équation d'induction ont été dérivées à partir d'un modèle bi-fluide afin d'étendre la loi exacte à d'autres gammes d'échelles et régimes. Les résultats obtenus ainsi ont servi de base afin d'adapter la loi exacte \cacro{CGL} aux modèles qui seront étudiés numériquement dans la Partie \ref{part_3}. Ces corrections serviront à refléter au mieux la cascade turbulente simulée.
