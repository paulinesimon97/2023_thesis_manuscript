Les fonctions de distribution de vitesse des ions observées dans le vent solaire sont généralement anisotropes le long des directions parallèle et perpendiculaires au champ magnétique [\cite{marsch_pronounced_1981}, \cite{matteini_evolution_2007}, \cite{bale_magnetic_2009}]. Le champ magnétique, interagissant avec les ions, rend le milieu anisotrope et les collisions, trop peu nombreuses, échouent à l'isotropiser. Ce type d'anisotropie a tout d'abord été modélisé par [\cite{chew_boltzmann_1956}] à travers une pression de forme tensorielle et diagonale (gyrotrope) et supposant l'isentropie du modèle. Ce modèle, nommé \cacro{CGL} en hommage aux auteurs, sera présenté plus en détail dans le Chapitre \ref{ch-21} de cette deuxième partie. Il y sera accompagné de l'extension proposée pour la théorie de Kolmogorov, prenant en compte un tenseur de pression. Ensuite dans le Chapitre \ref{ch-22}, nous nous poserons la question suivante : l'incompressibilité est-elle compatible avec la gyrotropie de pression ? Et, dans le Chapitre \ref{ch-23}, nous 
généraliserons la loi pour la \cacro{MHD} au modèle bi-fluide dépendant des ions et des électrons.

Dans cette partie qui concentre le cœur analytique du travail effectué, nous conserverons l'hypothèse d'une zone inertielle isentrope et nous ne regarderons pas en détail l'impact sur la cascade des composantes non gyrotropes du tenseur de pression.
