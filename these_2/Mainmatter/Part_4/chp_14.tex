Au cours de ce travail, nous avons questionné l'effet de la pression sur la cascade turbulente à travers l'extension de la théorie des lois exactes de Kolmogorov et la dérivation d'une telle loi pour la cascade d'énergie totale. Deux formats de la loi sont possibles, le format \cacro{KHM} qui ne prend en compte que l'hypothèse d'homogénéité statistique et le format \cacro{K41} qui implique la validité des hypothèses de stationnarité statistique et de séparation des échelles de forçage et de dissipation. Les échelles où la loi exacte de format  \cacro{K41} est constante sont les échelles inertielles. Dans la partie introductive, nous avons rappelé qu'un plasma peut être décrit tel un fluide grâce à un ensemble infini d'équations dépendant les unes des autres. La pression y apparaît sous la forme d'un tenseur dont l'équation dépend du flux de chaleur. Redéfinir une pression ou un flux de chaleur est une manière de fermer un tel système. 

\section{Partie \ref{part_1}}

Dans la Partie \ref{part_1}, la pression est supposée isotrope. L'utilisation d'une pression isotrope appelle en général une fermeture provenant de la thermodynamique (comme cela est décrit dans le Chapitre \ref{ch-12}). Diverses fermetures sont possibles (isotherme, adiabatique, isobare, isochore...) et il en existe une, un peu plus générale : la fermeture polytrope. En fonction de deux paramètres ($\sigma$ et $\gamma$), elle permet de se placer dans le cadre des autres fermetures et de décrire divers plasmas spatiaux. À partir de ces fermetures, on peut estimer une expression de l'énergie interne du système. Ces étapes sont usuellement effectuées avant la dérivation d'une loi exacte associée au système d'équations qui est ainsi fermé. 

Dans le Chapitre \ref{ch-13}, cette procédure est revisitée. En effet, l'expression explicite de l'énergie interne ou de la pression peut n'être appliquée qu'à la fin, après avoir obtenu une loi exacte que l'on peut ainsi qualifier de générale. La dérivation de cette loi ne nécessite que les équations de continuité, de vitesse, d'induction et d'énergie interne. La seule contrainte imposée sur la fermeture sera donc qu'elle respecte l'équation d'énergie interne. C'est le cas des fermetures thermodynamiques, l'équation d'énergie interne n'étant alors qu'une réécriture du premier principe de la thermodynamique. Afin de répondre à l'objectif initial qui était d'étendre la théorie des lois exactes à des écoulements polytropes, nous avons choisi d'appliquer la fermeture polytrope dans cette loi exacte générale. En supposant que le chauffage effectif du plasma impactera son entropie, nous avons supposé nulle la contribution du flux de chaleur dans la zone inertielle. Cette dernière décrite ainsi est isentrope. Cette hypothèse sur la zone inertielle est décorrélée du comportement thermodynamique global de l'écoulement. Si l'écoulement est adiabatique, la loi exacte obtenue sera valable en dehors de la zone inertielle, mais s'il est isotherme, par exemple, l'hypothèse isentrope s'accompagne du postulat que le système s'adaptera pour réguler la situation à l'extérieur de la zone inertielle. 

Dans le Chapitre \ref{ch-14}, une application de cette loi isentrope-polytrope est entreprise dans les données relevées par \cacro{PSP} dans le vent solaire. On y compare dans un jeu de données quasi-incompressible et un jeu plus compressible, le comportement des lois incompressible, isentrope-isotherme et isentrope-adiabatique, estimant ainsi le taux de cascade/chauffage dans le vent solaire. Ce taux s'avère dominé par les termes flux qui survivent dans la limite incompressible, dit Yaglom. La contribution de l'énergie interne, seule contribution dépendant de la fermeture calculable avec une seule sonde, reste négligeable. Le choix de fermetures thermodynamiques n'impacte alors pas le taux de chauffage contrairement à la présence de la densité dans les termes Yaglom. Ces observations ont ensuite été confirmées statistiquement dans les données \cacro{PSP} par \cite{brodiano_statistical_2022}. Une petite évaluation statistique est aussi menée dans les données relevées par \cacro{MMS} dans la magnétogaine plus compressible : le comportement de ces termes y est similaire. Étudier la cascade dans le vent solaire permet de confronter nos résultats analytiques à un écoulement réel, mais est aussi très contraignant dans l'étude des lois exactes compressibles : la plupart des missions envoyées ne comprenant qu'un seul satellite, il n'est alors pas possible de calculer des dérivées locales dans les données relevées. \cacro{MMS}, constellation de quatre satellites, permet par contre de les estimer et de calculer l'ensemble de la loi exacte. Ce travail n'a pas été entamé. Pour estimer l'énergie interne dans le vent solaire, nous avons utilisé la densité à travers une fermeture thermodynamique. Cette estimation, en accord avec le domaine de validité de la loi exacte appliquée, n'est pas très réaliste. Si l'on voulait être réaliste, il faudrait utiliser la pression issue des fonctions de distribution en vitesse des particules, or la pression dans le vent solaire n'est pas isotrope. Notre loi exacte n'est donc pas suffisante. 

\section{Partie \ref{part_2}}

Le vent solaire étant magnétisé, toutes les facettes de son comportement sont anisotropes suivant la direction du champ magnétique. La pression, second moment de la fonction de distribution en vitesse des particules, est ainsi impactée. Le vent solaire étant aussi peu collisionnel, cette anisotropie n'est pas homogénéisée par les collisions et la pression doit au minimum être décrite par un tenseur gyrotrope dépendant de deux composantes. La question de l'impact d'une telle propriété sur la cascade turbulente, cœur de cette thèse, est d'abord abordée à travers une extension analytique de la théorie des lois exactes, présentée dans la Partie \ref{part_2}. 

Comme résumé dans le Chapitre \ref{ch-21}, nous avons d'abord appliqué la routine analytique présentée dans le chapitre \ref{ch-13} et dépendant de l'équation d'énergie interne dans les équations dépendant d'une pression tensorielle. Puis nous y avons injecté la fermeture liée au modèle gyrotrope le plus communément utilisé : le modèle \cacro{CGL}. Ce modèle suppose nulle la contribution du flux de chaleur. Le résultat alors obtenu dans la zone inertielle décrit aussi le transfert non-linéaire de l'énergie à l'ensemble des échelles \cacro{MHD} en dehors de la zone inertielle. À travers la composante anisotrope du tenseur de pression, il dépend de l'anisotropie de pression mesurée usuellement par le rapport $a_p$. Cette contribution pourrait être affectée par le développement d'instabilité de pression telle que l'instabilité firehose ou l'instabilité miroir qui sont permises par l'anisotropie. Ce qui est certain, c'est qu'elle est la signature de l'impact de l'anisotropie de pression sur la cascade turbulente. 

En cherchant la limite incompressible de cette contribution, nous nous sommes apperçus qu'un terme parmi les quatre la composant survivait. Il est présenté dans le Chapitre \ref{ch-22}. L'anisotropie de pression étant généralement étudiée dans le cadre compressible, ce terme interpelle : gyrotropie et incompressibilité seraient-elles compatibles ? Nous proposons alors un modèle auto-cohérent dépendant d'un tenseur de pression gyrotrope, fermé par la contrainte incompressible et l'équation d'énergie interne indépendante du flux de chaleur. L'énergie interne étant définie à partir de la trace du tenseur de pression, cette fermeture semble appropriée pour s'assurer de la compatibilité de ce modèle avec notre loi exacte générale. L'étude linéaire de ce modèle a révélé la présence d'un mode d'Alfvén incompressible affecté par la correction firehose ainsi que d'un nouveau mode lié à la limite incompressible des modes magnétosonores du modèle \cacro{CGL}. La cascade turbulente étant en partie développée par des interactions non-linéaires entre ondes, notre correction gyrotrope de la loi \cacro{MHD} incompressible pourrait être l'incarnation de ces interactions. De plus, la résilience de ce terme dans la limite incompressible semble impliquer que la correction majeure dans le vent solaire, quasi-incompressible, n'est peut-être pas la compressibilité mais l'anisotropie de pression. 

Dans le Chapitre \ref{ch-23}, nous sortons du cadre \cacro{MHD} en vérifiant le comportement des différents termes de la loi d'Ohm face à l'utilisation de tenseurs de pression. Le terme Hall n'en dépendant pas, sa contribution à la loi exacte ne varie pas entre le cas d'une pression isotrope ou tensorielle. Par contre le terme \cacro{Pe}, en fonction de la forme du tenseur de pression électronique, pourrait influer le transfert non-linéaire, et cela même dans la limite incompressible. Nous y dérivons aussi une loi exacte générale bi-fluide dépendant des ions et des électrons et valable dans divers régimes : \cacro{MHD} et \cacro{EMHD}. Ces extensions permettent la description du transfert non-linéaire dans la grande majorité des gammes d'échelles mesurables. Notons tout de même que la formulation de la loi exacte ainsi obtenue est intrinsèque à la fonction de corrélation d'énergie totale choisie au début le calcul. L'analyse des contributions non gyrotropes du tenseur de pression n'a pas été abordée, mais notre loi générale dépendant d'une pression tensorielle semble tout à fait compatible avec leur étude (tant que le tenseur de pression reste symétrique).   

\section{Partie \ref{part_3}}
La description de la cascade turbulente d'énergie totale, étendue à ce qu'il semble être son maximum (la contribution du flux de chaleur a été dérivée dans le Chapitre \ref{ch-13}), peut maintenant nous permettre d'étudier dans son intégralité la cascade d'énergie présente dans des simulations \sacro{LFCGLHPe} et en particulier l'effet de l'anisotropie de pression. Cette étude entamée et présentant des premiers résultats prometteurs est décrite dans la Partie \ref{part_3}. 

Le Chapitre \ref{ch-31} présente le code versatile permettant de simuler divers modèles dépendant d'une pression gyrotrope tel que le modèle \sacro{CGL} ou le modèle \sacro{LF}. Ce code peut être utilisé afin de répondre aux problématiques concernant la turbulence grâce à l'inclusion d'un forçage et de l'hyperdissipation, originellement utilisée pour lisser les fort gradients pouvant induire des instabilités numériques, mais qui s'avère être un canal dissipatif performant à petites échelles dans le cadre des études de turbulence. Y sont aussi présentés notre code de post-traitement et les choix imposés afin de visualiser les résultats. La particularité principale de notre code de post-traitement est qu'il repose sur le lien entre fonction de corrélation et produit de convolution. Cette propriété mathématique nous permet d'effectuer le calcul des termes de la loi exacte dans l'espace de Fourier, et nous donne la quantité voulue pour l'ensemble des vecteurs d'échelle accessibles en fonction de la taille et de la résolution de la simulation traitée. Le résultat final, tridimensionnel et régulier, laisse ainsi libre les choix de traitements supplémentaires (dérivation, etc.) et de réduction (carte bidimensionnelle, filtrage angulaire et moyenne pour obtenir une visualisation \cacro{1D}, etc.) afin de visualiser les comportements turbulents qui nous intéressent. 

Une campagne de validation de notre implémentation est présenté dans le Chapitre \ref{ch-32} à travers les comparaisons de résultats de diverses lois exactes et une estimation de l'erreur numérique sur le taux de cascade totale, erreur intrinsèque à nos simulations. Cette erreur est obtenue en calculant la forme \cacro{KHM} de nos lois exactes, elle prend donc en compte les spécificités du code initial (forçage et hyperdissipation) et son estimation n'est permise que grâce à l'extension de la théorie des lois exactes présentée dans les Parties \ref{part_1} et \ref{part_2}. Nous intéresser au comportement des termes de forçage et d'hyperdissipation nous a permis de nous rendre compte des limites mathématiques de cette théorie reposant sur des fonctions de corrélation d'ordre 2. En effet, le comportement de l'hyperdissipation par exemple n'est pas calqué sur sa conception dans l'espace spectral, mais sur des saturations mathématiques dont les tendances dans le cas isotrope sont estimées dans l'Annexe \ref{an:A}. 

Ayant ainsi validé et déterminé les limitations de notre code, nous avons entamé l'étude des simulations dont certaines sont utilisées par \cite{ferrand_fluid_2021} du modèle \cacro{CGL}. Cette étude fait l'objet du Chapitre \ref{ch-33}. Le comportement des premiers résultats étant particulier, il a engendré un certain nombre de questions qui ont motivé le lancement de nouvelles simulations. L'interprétation fine de leurs résultats n'est pas encore aboutie. Cependant, il semblerait que dans ces simulations quasi-incompressibles, le terme survivant à la limite incompressible domine les termes compressibles. Cela irait dans le sens de la conjecture proposée dans la Partie \ref{part_2} : dans un plasma quasi-incompressible (par exemple le vent solaire), la correction de l'anisotropie de pression primerait sur la correction compressible de la loi exacte et de l'estimation du taux de chauffage turbulent. Une étude dans les données de \cacro{MMS} ou de la mission \sacro{HELIOSWARM} qui sera déployée dans le vent solaire en 2028 pourrait permettre de valider ce résultat, notre contribution dépendant de dérivées locales. En attendant, nos résultats analytiques pourraient aussi permettre de mieux comprendre la cascade turbulente dans les modèles \cacro{LF} tel qu'ébauché dans le Chapitre \ref{ch-34}.

\section{Le mot de la fin du début}
Finalement, cette thèse revisite diverses méthodes liées à l'étude de la cascade turbulente telle que la méthode d'obtention des lois exactes d'énergie totale ou leur implémentation en tant que post-traitement de simulations turbulentes. Elle apporte aussi un cadre d'étude élargie de la cascade d'énergie totale grâce à une extension généraliste de la théorie des lois exactes, un nouveau modèle incompressible gyrotrope ainsi qu'une correction dépendant des anisotropies de pression. Cette correction pourrait servir de base à l'étude du lien entre les instabilités du pression et la turbulence, et semble être plus importante dans la description d'un écoulement quasi-incompressible que la correction donnée par la compression. 

Ces apports pourront servir de base pour des études plus approfondies de la turbulence dans les plasmas spatiaux. Jusqu'à présent, dans les données relevées dans le vent solaire par exemple, on n'utilise pas la pression pour calculer les lois exactes. On ne prend en compte que la densité, la vitesse et le champ magnétique et on calcule une pression équivalente grâce à une fermeture, comme effectué dans le Chapitre \ref{ch-14}. La loi exacte non fermée dépendant des tenseurs de pression nous offre maintenant plus de liberté. Pour ce qui est de la correction dépendant du flux de chaleur, elle sera, pour l'instant, difficile à calculer, l'extraction du flux de chaleur des données étant limitée à cause des précisions des coupes de Faraday utilisées. Pour le calcul de termes sources, dont notre terme correctif dominant, \sacro{HELIOSWARM} (9 sondes) pourrait permettre, dans un futur proche, de les étudier plus précisément dans le vent solaire. Cependant, une amélioration de la précision des techniques de calcul des gradients locaux sera sûrement nécessaire. 

Jusqu'à présent, ce travail a fait l'objet de deux articles. 
\begin{itemize}
 \item Le premier, \cite{simon_general_2021}, résume l'obtention de l'extension de la théorie des lois exactes dépendant d'une pression isotrope dans le cadre d'une description isentrope de la zone inertielle et présente les résultats de l'étude de cas effectuée avec les données de \cacro{PSP}.
 \item Le second, \cite{simon_exact_2022}, présente l'extension analytique dépendant de la compression et d'une pression tensorielle avec l'application au modèle \cacro{CGL} et sa limite incompressible.
\end{itemize}
  Ils sont insérés à la fin de cette thèse. D'autres articles sont en préparation sur le modèle incompressible gyrotrope, l'étude des simulations et les autres extensions abordées dans cette thèse.    

